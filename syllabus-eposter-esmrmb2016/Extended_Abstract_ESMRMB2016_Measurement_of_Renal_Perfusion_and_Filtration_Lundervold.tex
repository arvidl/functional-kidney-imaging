% Length of syllabus contributions should be one page or more.  Submission should be via message with attached file sent to syllabus11@ismrm.org . 
% Syllabus contributions will be published in the weekend syllabus CD.    
%
% The syllabus will be published (and made available on line to pre-registrants) two weeks before the meeting.  
% Please use the following guidelines in developing your contribution:
%
% 1.         Paper size:  Should be formatted for 8.5� x 11� or A4 paper.
% 2.         Format:  Single spaced in 10 or 12 pt type; Margins:  At least 1� or 25.4 mm margins on all sides.
% 3.         Use numerals in brackets for footnote or reference notations.


\documentclass[11pt]{article}
\usepackage[left=2.0cm,top=1.0cm,right=2.5cm,bottom=2.0cm]{geometry}
\usepackage{graphicx}
\usepackage{amssymb}
\usepackage{times}
\usepackage{float}
\usepackage{multicol}
\usepackage{url}
\usepackage{comment}
\usepackage{enumitem}


\title{{\small SYLLABUS} \\Measurement of Renal Perfusion and Filtration \vspace{-3mm}}
\author{Arvid Lundervold, {\small BSc, MD, PhD} \\ {\small Department of Biomedicine, University of Bergen \& Department of Radiology, Haukeland University Hospital, Bergen, Norway} \\ {\vspace{-2mm}\footnotesize \tt http://computationalmedicine.no}}
\date{ESMRMB 2016 - 33$^{\mbox{rd}}$ Annual Scientific Meeting,
Vienna, Austria\\ Teaching Session: Functional kidney imaging with MRI (adv.) \\ {\small Thursday 29-Sep-2016, 0800-0830}\\{\footnotesize \tt https://github.com/arvidl/functional-kidney-imaging}}                                           

\begin{document}
\maketitle
%\section{}

\subsection*{Abstract}

Kidney function is related to maintenance of fluids, electrolytes, acid-base balance and clearance of toxins. Normal kidney function is maintained by 
coordinated regulation at different levels of organization warranting an integrative approach to kidney function in health and disease.

Important parameters describing kidney physiology and function are: {\bf renal plasma flow}, RPF = (1 - Hct) $\cdot$ RBF, where Hct (hematocrit) is the fraction 
of blood volume representing the cellular elements of blood, RBF is the volume of blood delivered to the kidneys per unit time [mL/min]; {\bf renal perfusion}, 
denoting renal blood flow per unit volume of kidney [mL/min/100 mL]; and {\bf glomerular filtration rate}, GFR, the volume of fluid filtered from the renal 
glomerular capillaries into Bowman's capsule per unit time, equal to the sum of the filtration rates of all functioning nephrons [mL/min]. Typical values in 
healthy humans: Hct $\sim$0.40, RPF $\sim$600 mL/min, GFR $\sim$125 mL/min (70-kg man, where population studies have shown GFR to be proportional to body 
surface area, and sex and age dependent).

This teaching session will introduce measurement of renal perfusion and filtration in the framework of {\bf imaging-based biomarkers} 
and {\bf computational medicine} / {\bf systems medicine}.\\

\noindent ({\bf i}) The first part will give a brief overview of kidney structure and function and the key physiological parameters relevant to the clinics. 
In particular, we describe the gross functional anatomy of the kidney, the renal blood supply, and the ultrastructure of the filtration barrier, as well 
as some major diseases and conditions affecting normal perfusion and filtration. To complete this motivational part, we mention assessment 
of kidney function in the clinical laboratory based on the measurement of the clearance of various substances (exogenous and endogenous markers) 
by the kidneys, incorporating the `{\bf conservation of mass}' principle also underlying imaging-based modeling of renal perfusion and filtration.\\

\noindent ({\bf ii}) The second part addresses MR acquisition techniques in use for measuring perfusion and filtration, with a focus on DCE-MRI.\\

\noindent ({\bf iii}) This part presents {\bf tracer kinetics} and {\bf compartment models} applied to parenchymal regions down to single voxels, where 
{\bf motion correction} can be an issue.\\

\noindent ({\bf iv}) In the fourth part, model-based estimation of renal perfusion and filtration using different numerical {\bf software} and programming languages 
will be demonstrated.\\

\noindent Finally, we will point to relevant literature, software tools, and new initiatives addressing the lack of standardization in acquisition and analysis methods, 
limited access to data from previous studies, and the challenges of {\bf reproducibility} and {\bf validation}.


\subsection*{Extended list of bibliographical references:}

\vspace{2mm}

\begin{description}[style=unboxed,leftmargin=0.1cm]

\item[{\rm \underline{Kidney function}}]:  \\  {\footnotesize\cite{Artunc2011,Bennett2013,Bertram2014,Carlstroem2015,Ebrahimi2014,Edwards2010,Evans2013,Grenier2003,Grenier2011,Grenier2016,Hausmann2012,Jones2011,Koeppen2013,Korhonen2015,Levey2014,Michaely2007,Murphy2014,Murray2016,Layton2013,Layton2014,Mazza2010,Pallone2012,Pohlmann2013,Rahn1999,Rusinek2004,Thomas2009,Thomson2012,Turner2015,Zhang2013}}

\vspace{2mm}

\item[{\rm \underline{Computational physiology / medicine}}]:  \\  {\footnotesize \cite{Clegg2015a,Guan2015,Gupta2013,Harder2015,Harris2009,He2012,Hunt2016,Kretzler2015,Mariani2015,Meghdadi2016,Moss2009,Neusser2008,Pannabecker2014,Perco2010,Sgouralis2015,Stegall2015,Winslow2012}}

\vspace{2mm}


\item[{\rm \underline{Imaging-based biomarkers}}]:  \\  {\footnotesize \cite{WikipediaBiomarker,Abramson2015,Buckler2011,Cutajar2015,Emre2016,Haq2015,Herrmann2016,Kessler2015,Kline2016,Kontopodis2015,Mirka2015,Notohamiprodjo2010,Obuchowski2015,Obuchowski2016,Pedrosa2009,Piludu2015,Rapacchi2015,Raunig2015,Roseman2016,Rosenkrantz2015,Sterzik2015,Sullivan2015,Sweis2016,Takahashi2015,Taouli2016,Wu2015,Yang2015,Zoellner2016a}}


\vspace{2mm}
 
\item[{\rm \underline{DCE-MRI}}]:  \\   {\footnotesize\cite{Bane2016,Barnes2014a,Chen2015,DeNaeyer2011,Dickie2016,Eikefjord2016a,Han2016,Li2016,Kratochvila2016,Lietzmann2012,Mehrtash2016,Notohamiprodjo2011,Rajan2014,Schabel2008,Seo2016,Simonis2016,Sourbron2013,Subashi2014,vanSchie2015,Wright2014,Xie2016,Zhang2015,Zhang2015a}}
 
\vspace{2mm} 

\item[{\rm \underline{BOLD fMRI}}]: \\   {\footnotesize\cite{Cantow2016,Chrysochou2012,Donati2012,Ebrahimi2012,Gloviczki2013,Inoue2012,Jahanian2014,Khatir2015,Lerman2014,Li2004,Li2008,Michaely2012,Neugarten2012,Neugarten2014,Niendorf2015,Niles2016,Notohamiprodjo2013,Park2012,Piskunowicz2015,Prasad1999,Prasad2006,Pruijm2013,Pruijm2014,Saad2013,Thacker2015,Vink2015a,Zhang2014,Zheng2014}}

\vspace{2mm} 

\item[{\rm \underline{Tracer kinetics \& compartment modeling}}]: \\  {\footnotesize \noindent \cite{Brix2010,Chen2011a,Chen2014,Cutajar2010,Gill2015,Ingrisch2013,Kallehauge2016,Khalifa2014,Koh2011,Liberman2016,Lee2014,Matis1980,Michoux2006,Nadav2016,Roberts2011,Sommer2014,Sourbron2008,Sourbron2011,Sourbron2012,Taheri2016,Tofts1999,Tofts2012,Turco2016,Wang2016,Winter2014}}

\vspace{2mm}

\item[{\rm \underline{Motion correction}}]:  \\  {\footnotesize \cite{Attenberger2010,Buonaccorsi2007,deSenneville2008,DenisdeSenneville2015,El-Baz2006,El-Baz2007,Gardener2010,Hodneland2014,Hodneland2014b,Kalis2016,Liu2014,Merrem2013,Positano2013,Song2006,Yang2012,Zoellner2009}}

\vspace{2mm}

\item[{\rm \underline{Kidney segmentation}}]: \\   {\footnotesize \cite{Akbari2013,Chiusano2014,El-Baz2005,Gloger2012,Gloger2015,Hanson2013,Karstoft2007,Khalifa2013,Kim2016,Li2012,Luo2013,Rusinek2007,Rusinek2016,Will2014,Woodard2015,Yang2016,Zhang2016b,Zhou2013,Zoellner2012}}
  
\vspace{2mm}

\item[{\rm \underline{Validation, accuracy \& reproducibility}}]: \\  {\footnotesize \cite{Annet2004,Artz2011,Bosca2016,Braunagel2015,Buckley2006,Cutajar2014,Eikefjord2015,Eikefjord2016,Hammon2016,Khatir2014,Lim2013,Luedemann2009,Mendichovszky2008,Mendichovszky2009,Notohamiprodjo2013,Wang2015,Warmuth2007,Winter2011,Zhang2016a,Zimmer2013}}
  
\vspace{2mm}

\item[{\rm \underline{Software tools {\small (DCE-MRI, visual analytics)}}}]: \\ {\footnotesize\cite{Angulo2016,Avants2011,Barnes2015,Beuzit2016,Chen2011,Ferl2011,Lee2009,Ortuno2013,Schmid2006,Schmid2009,Schmid2009a,Smith2015,Sourbron2009,Zoellner2013,Zoellner2016}}
 

\end{description}

 
 \vspace{-10mm}
 
{\small

%\renewcommand\refname{\large Suggested reading list}
\renewcommand\refname{ }

%\begin{multicols}{1}
\bibliographystyle{bmc_article}
%\bibliographystyle{plain}
%\bibliographystyle{unsrt}
%\bibliographystyle{plain-annote}
%\bibliographystyle{IEEEannot}
\bibliography{mr_renography}
%\end{multicols}

}


\end{document}  